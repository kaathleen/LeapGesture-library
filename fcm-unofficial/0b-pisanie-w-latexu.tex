
\chapter{Składanie dokumentu w systemie \LaTeX}

Po pierwsze to gratulacje --- dobry wybór. W tym rozdziale znajduje się
garść informacji o tym, jak poprawnie składać tekst pracy w systemie \LaTeX{} wraz z 
przykładami, które mają służyć do przeklejania do własnych dokumentów.

\section{Narzędzia}
Pracując pod systemem Windows, polecam:
\begin{itemize}
    \item MikTeX, \url{http://www.miktex.org/},
    \item JEdit, \url{http://www.jedit.org/},
    \item TeXlipse, \url{http://texlipse.sourceforge.net/},
    \item Kile, \url{http://kile.sourceforge.net/},
    \item Ghostview, Ghostscript (podgląd dokumentów PDF bez blokowania pliku):\\
        \url{http://www.cs.wisc.edu/~ghost/}. 
\end{itemize}

Po zainstalowaniu tych narzędzi wystarczy wykonać polecenie \texttt{compile.bat} (który
jest skryptem wsadowym dla Windows). Dla tych, którzy wolą nieco automatyzacji --- skrypt
\texttt{latexmk}, który jest w MikTeXu (a który potrzebuje zainstalowanego Perla) jest
również bardzo wygodny: \texttt{latexmk -pdf -pvc main.tex}.

\section{Edycja tekstu}
\chaptermark{Tytuł rozdziału, jeśli pełen się nie mieści\ldots{}}{}

\subsection{Struktura dokumentu}

Praca składa się z rozdziałów (\texttt{chapter}) i podrozdziałów (\texttt{section}).
Ewentualnie można również rozdziały zagnieżdzać (\texttt{subsection}, \texttt{subsubsection}),
jednak nie powinno się wykraczać poza drugi poziom hierarchii (czyli \texttt{subsubsection}).

\subsection{Akapity i znaki specjalne}

Każdy akapit to po prostu blok tekstu. Nieważne jak sformatowany --- to zrobi już
system $\LaTeX$.

Akapity rozdziela się od siebie przynajmniej jedną pustą linią. Podstawowe
instrukcje, które się przydają to \emph{wyróżnienie pewnych słów}. Można również
stosować \textbf{styl pogrubiony}, choć nie jest to generalnie zalecane.

Należy pamiętać o zasadach polskiej interpunkcji i ortografii. Po spójnikach 
jednoliterowych warto wstawić znak tyldy ($\sim$), który jest tak zwaną
,,twardą spacją'' i powoduje, że wyrazy nią połączone nie będą rozdzielane
na dwie linie tekstu.

Polskie znaki interpunkcyjne różnią się nieco od angielskich: to jest ,,polski'', a to jest
``angielski''. W kodzie źródłowym tego tekstu będzie widać różnicę.

Proszę również zwrócić uwagę na znak myślnika, który może być pauzą ,,---'' lub
półpauzą: ,,--''. Należy stosować je konsekwentnie. Do łączenia wyrazów używamy
zwykłego ,,-'' (\emph{północno-wschodni}), do myślników --- pauzy lub półpauzy.
Inne zasady interpunkcji i typografii można znaleźć w słownikach.

\subsection{Wypunktowania}

Wypunktowanie z cyframi:
\begin{enumerate}
    \item to jest punkt,
    \item i to jest punkt,
    \item a to jest ostatni punkt.
\end{enumerate}

\noindent
Po wypunktowaniach czasem nie warto wstawiać wcięcia akapitowego. Wtedy przydatne jest
polecenie \texttt{noindent}. Wypunktowanie z kropkami (tzw.~\emph{bullet list}) wygląda tak:
\begin{itemize}
    \item to jest punkt,
    \item i to jest punkt,
    \item a to jest ostatni punkt.
\end{itemize}

\noindent
Wypunktowania opisowe właściwie niewiele się różnią:
\begin{description}
    \item[elementA] to jest opis,
    \item[elementB] i to jest opis,
    \item[elementC] a to jest ostatni opis.
\end{description}


\subsection{Polecenia pakietu \texttt{ppfcmthesis}}

Parę poleceń zostało zdefiniowanych aby uspójnić styl pracy. Są one przedstawione poniżej
(oczywiście nie trzeba się do nich stosować).

\paragraph{Makra zdefiniowane dla języka angielskiego.} Są nimi: \texttt{termdef} oraz \texttt{acronym}.
Przykłady poniżej obrazują ich przewidywane użycie w tekście.
\begin{center}\footnotesize%
\begin{tabular}{l >{\rightskip\fill}p{12cm}}
\toprule
źródło   & \texttt{we call this a $\backslash$termdef\{Database Management System\} ($\backslash$acronym\{DBMS\})} \\ \cmidrule(lr){2-2}
docelowo & we call this a \termdef{Database Management System} (\acronym{DBMS}) \\ 
\bottomrule
\end{tabular}
\end{center}

\paragraph{Makra zdefiniowane dla języka polskiego.} Podobnie jak dla języka angielskiego zdefiniowano
odpowiedniki polskie: \texttt{defini\-cja}, \texttt{akronim} oraz \texttt{english} dla tłumaczeń angielskich
terminów. Przykłady poniżej obrazują ich przewidywane użycie w tekście.
\begin{center}\footnotesize%
\begin{tabular}{l >{\rightskip\fill}p{12cm}}
\toprule
źródło   & \texttt{nazywamy go $\backslash$definicja\{systemem zarządzania bazą danych\} ($\backslash$akronim\{DBMS\}, $\backslash$english\{Database Management System\})} \\ \cmidrule(lr){2-2}
docelowo & nazywamy go \definicja{systemem zarządzania bazą danych} (\akronim{DBMS}, \english{Database Management System}) \\ \bottomrule
\end{tabular}
\end{center}


\subsection{Rysunki}

Format wstawianych rysunków zależy od tego czy używa się do kompilacji polecenia
\texttt{latex}, czy też \texttt{pdflatex}. Oba powinny dać dokładnie ten sam wynik końcowy,
ale praca z nimi jest nieco inna.

\begin{description}
    \item[latex] To polecenie kompiluje źródła \LaTeX{}owe do pliku 
        z rozszerzeniem \texttt{dvi}. Ten plik można przeglądać przy pomocy specjalizowanych programów
        takich jak przykładowo Yap obecny z dystrybucją Mik\TeX{}a. Aby uzyskać docelowy plik \akronim{PDF}
        należy przekonwertować plik \texttt{dvi} przy pomocy programu \texttt{dvipdfm}. 
        
        \textbf{UWAGA:} korzystając z programu \texttt{latex}, wszystkie rysunki muszą być w formacie \akronim{EPS}
        (\english{encapsulated postscript}).

    \item[pdflatex] To polecenie kompiluje źródła \LaTeX{}owe bezpośrednio do pliku \akronim{PDF}.
    
        \textbf{UWAGA:} korzystając z programu \texttt{pdflatex}, wszystkie rysunki muszą być w formacie \akronim{PDF},
        \akronim{JPG} lub \akronim{PNG}.
\end{description}

Można oczywiście używać obu systemów --- wtedy pliki rysunków muszą po prostu być dostępne w obu formatach.

Wszystkie rysunki (w tym również diagramy, szkice i inne) osadzamy w środowisku 
\texttt{figure} i umieszczamy podpis \emph{pod} rysunkiem, w formie elementu \texttt{caption}. Rysunki powinny
zostać umieszczone u góry strony (osadzone bezpośrednio w treści strony zwykle utrudniają czytanie tekstu).
Rysunek~\ref{rys:plama} zawiera przykład pełnego osadzenia rysunku na stronie.

\begin{figure}[t] % możliwe opcje to 't' - top, 'b' - bottom, 'h' - 'here', ale zaleca się 't'
\centering\includegraphics[width=5cm]{figures/template/logo-pp}
\caption{Logo Politechniki Poznańskiej.}\label{rys:plama}
\end{figure}

\begin{figure}[t]
\centering\includegraphics[width=5cm]{figures/template/logo-pp}
\fcmfcaption{Logo Politechniki Poznańskiej. Formatowanie zgodne z wytycznymi FCMu.}\label{rys:plama2}
\end{figure}

Zasady FCMu sugerują nieco inne nagłówki rysunków. Dostepne są one poleceniem \texttt{fcmfcaption} (zob.~rysunek
\ref{rys:plama2}), jeśli ktoś woli mieć podpisy niespójne z rysunkami\ldots

\subsection{Tablice}

Tablice to piękna rzecz, choć akurat ich umiejętne tworzenie w \LaTeX{}u nie jest łatwe. 
Jeśli tablica jest skomplikowana, to pewnie łatwiej będzie ją wykonać w programie
OpenOffice, a następnie wyeksportować jako plik \akronim{PDF}. W każdym przypadku tablice wstawia się podobnie
jak rysunki, tylko że w środowisko \texttt{table}. Tradycja typograficzna sugeruje umieszczenie opisu tablicy, a więc
elementu \texttt{caption} ponad jej treścią (inaczej niż przy rysunkach).  

Tablica~\ref{tab:tabela} pokazuje pełen przykład.

\begin{table}[h]
\caption{Przykładowa tabela. Styl opisu jest zgodny z rysunkami.}\label{tab:tabela}
\centering\footnotesize%
\begin{tabular}{l c}
\toprule
artykuł & cena [zł] \\
\midrule
bułka   & $0,4$ \\
masło   & $2,5$ \\
\bottomrule
\end{tabular}
\end{table}

Zasady FCMu sugerują nieco inne nagłówki tablic. Dostepne są one poleceniem \texttt{fcmtcaption} (zob.~tablicę
\ref{tab:tabela2}), jeśli ktoś woli mieć podpisy niespójne z rysunkami\ldots

\begin{table}[h]
\fcmtcaption{Przykładowa tabela. Styl opisu jest zgodny z wytycznymi FCMu.}\label{tab:tabela2}
\centering\footnotesize%
\begin{tabular}{l c}
\toprule
artykuł & cena [zł] \\
\midrule
bułka   & $0,4$ \\
masło   & $2,5$ \\
\bottomrule
\end{tabular}
\end{table}


\subsection{Checklista}

W katalogu źródeł stylu \texttt{ppfcmthesis} znajduje się plik \texttt{CHECKLIST} --- należy
sprawdzić, czy nie popełniło się któregoś z wymienionych tam błędów.


\section{Literatura i materiały dodatkowe}

Materiałów jest mnóstwo. Oto parę z nich:
\begin{itemize}
    \item \emph{The Not So Short Introduction\ldots}, która posiada również tłumaczenie 
    w języku polskim.\\
    \url{http://www.ctan.org/tex-archive/info/lshort/english/lshort.pdf}

    \item Klasy stylu \texttt{memoir} posiadają bardzo wiele informacji o składzie tekstów
    anglosaskich oraz sposoby dostosowania \LaTeX{}a do własnych potrzeb.\\
    \url{http://www.ctan.org/tex-archive/macros/latex/contrib/memoir/memman.pdf}
    
    \item Nasza grupa dyskusyjna i repozytorium SVN są również dobrym miejscem aby zapytać
    (lub sprawdzić czy pytanie nie zostało już zadane).\\
    \url{https://ophelia.cs.put.poznan.pl/svn/put-latex/trunk}

    \item Dla łaknących więcej wiedzy o systemie \LaTeX{} podstawowym źródłem informacji
    jest książka Lamporta~\cite{Lamport:LDP85}. Prawdziwy \emph{hardcore} to oczywiście
    \emph{The \TeX{}book} profesora Knutha~\cite{Knuth:ct-a}.
\end{itemize}

