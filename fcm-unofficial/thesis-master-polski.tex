
% Szkielet dla pracy pisanej w języku polskim.

\documentclass[polish,a4paper,oneside]{ppfcmthesis}


\usepackage[utf8]{inputenc}
\usepackage[OT4]{fontenc}


\authortitle{}                                        % You can place "inż.~" here, if you really want to.
\author{Ignacy Iksiński}                              % Your name comes here
\title{W zdrowym ciele zdrowy~duch}                   % Note how we protect the final title phrase from breaking
\ppsupervisor{prof.~dr hab.~inż.~Alojzy Wołodyjowski} % Your supervisor comes here.
\ppyear{2006}                                         % Year of final submission (not graduation!)


\begin{document}

% Front matter starts here
\frontmatter\pagestyle{empty}%
\maketitle\cleardoublepage%

% Blank info page for "karta dyplomowa"
\thispagestyle{empty}\vspace*{\fill}%
\begin{center}Tutaj przychodzi karta pracy dyplomowej;\\oryginał wstawiamy do wersji dla archiwum PP, w pozostałych kopiach wstawiamy ksero.\end{center}%
\vfill\cleardoublepage%

% Table of contents.
\pagenumbering{Roman}\pagestyle{ppfcmthesis}%
\tableofcontents* \cleardoublepage%

% Main content of your thesis starts here.
\mainmatter%
\chapter{Wprowadzenie}

% [DW] Cytowania normalnie, bez dodatkowych nawiasów; poprawiłem niektóre, zrób pozostałe.

Drzewa sufiksów oraz tablice sufiksów, czyli tablice leksykograficznie posortowanych sufiksów pewnego
ciągu symboli, znajdują wiele praktycznych zastosowań:

\begin{itemize}
    \item w~problemach przetwarzania tekstu, nazywanych żargonowo~\emph{stringology}, takich
    jak dopasowywanie, przeszukiwanie, wyszukiwanie powtarzających się podciągów 
    i~maksymalnych sekwencji~\cite{Manber90, gusfield} oraz
    wielu innych;

    \item w~bioinformatyce~\cite{abouelhoda-genome}, gdzie
    pytania dotyczące kodu genetycznego można sprowadzić do rozwiązania problemów operujących 
    na ciągach znaków;

    \item w~kompresji danych~\cite{BW}, gdzie wykorzystywane są do wyszukiwania powtarzających
    się sekwencji symboli i do obliczania transformacji Burrowsa-Wheelera 
    (\english{Burrows-Wheeler transform}, \akronim{BWT}), która jest krokiem wstępnym
    takich metod kompresji danych, jak np.~\texttt{bzip2}.
\end{itemize}

Większość z~powyższych problemów daje się efektywnie rozwiązać przy pomocy 
drzew sufiksów (\english{suffix tree}). Udowodniono jednakże~\cite{replacing},
że tablice sufiksów są w~wielu przypadkach równie dobrą strukturą danych co drzewa sufiksów, 
a~ponieważ ich konstrukcja jest zwykle obarczona
mniejszym kosztem pamięciowym i~czasowym (mimo tej samej teoretycznej
złożoności $\bigO(n)$), to właśnie tablice sufiksów cieszą się coraz większym
powodzeniem w~praktyce.

Istnieje wiele algorytmów tworzenia tablic sufiksów, w~większości opublikowanych wraz z~implementacjami
w~języku \texttt{C} lub \texttt{C++}. O~ile od strony teoretycznej
większość z~tych algorytmów jest efektywna (liniowa), o~tyle w~praktyce ich efektywność zależy od aspektów
dostępu do pamięci oraz wielkości słownika symboli. Brakuje również
implementacji tych algorytmów w~językach wysokiego poziomu, takich jak język Java. 
Tym brakiem właśnie motywujemy potrzebę opracowania w~języku Java biblioteki algorytmów tworzenia 
tablic sufiksów. Przeniesienie samych algorytmów do języka Java wymaga zwrócenia szczególnej uwagi na
jego cechy specyficzne w~stosunku do języków niskiego poziomu -- różnice w~dostępie do pamięci 
(brak wskaźników, relokowalne struktury danych), zarządzanie pamięcią przez maszynę
wirtualną (\emph{garbage collector}), czy też fakt, że kod programu poddawany jest bezustannej obserwacji
i~kompilacji w~czasie rzeczywistym (\emph{just in time compilation}).


\section{Cel i~zakres pracy}

Głównym celem tej pracy jest \textbf{wybór najlepszych algorytmów tworzenia tablic sufiksów oraz
ich  efektywna implementacja w~postaci biblioteki  napisanej w~języku Java}.  W~zakres pracy wliczony jest również przegląd obecnie dostępnej literatury poświęconej  metodom
tworzenia tablic sufiksów oraz opis najciekawszych spośród nich. Docelowo powinna powstać
więc pewna taksonomia opierająca się na schemacie działania algorytmów i~umożliwiająca ich porównanie i~odniesienie względem siebie. 

Drugim celem pracy jest dokładne przetestowanie i~analiza utworzonej implementacji i~określenie jej wydajności na różnych
maszynach wirtualnych oraz na różnych platformach sprzętowych. Uzyskane wyniki eksperymentalne
skonfrontowane zostaną z~istniejącymi wynikami testów wydajnościowych implementacji tych
samych algorytmów w~językach \texttt{C} i~\texttt{C++}. 

\section{Struktura pracy}

W~rozdziałach drugim i~trzecim przedstawiono przegląd literatury dziedzinowej; rozdział drugi
zawiera podstawy teoretyczne wprowadzające czytelnika w~tematykę tablic sufiksów, a~trzeci
poświęcony jest klasyfikacji i~przeglądowi metod tworzenia tablic sufiksów. Rozdział czwarty zawiera
opisy wybranych algorytmów. W rozdziale piątym podane są wyniki empirycznych testów 
wydajnościowych wykonanych w różnych środowiskach i na różnych maszynach wirtualnych. 
Rozdział szósty stanowi podsumowanie pracy.



\chapter{Podstawy teoretyczne}

Rozdział teoretyczny --- przegląd literatury naświetlający stan wiedzy na dany temat. 

Przegląd literatury naświetlający stan wiedzy na dany temat obejmuje rozdziały pisane na podstawie
literatury, której wykaz zamieszczany jest w części pracy pt.~\emph{Literatura} (lub inaczej \emph{Bibliografia},
\emph{Piśmiennictwo}). W tekście pracy muszą wystąpić odwołania do wszystkich pozycji zamieszczonych w
wykazie literatury. \textbf{Nie należy odnośników do literatury umieszczać w stopce strony.} Student jest
bezwzględnie zobowiązany do wskazywania źródeł pochodzenia informacji przedstawianych w pracy,
dotyczy to również rysunków, tabel, fragmentów kodu źródłowego programów itd. Należy także podać
adresy stron internetowych w przypadku źródeł pochodzących z Internetu.




\chapter{Rozwinięcie}

Rozdziały dokumentujące pracę własną studenta: opisujące ideę, sposób lub metodę 
rozwiązania postawionego problemu oraz rozdziały opisujące techniczną stronę rozwiązania 
--- dokumentacja techniczna, przeprowadzone testy, badania i uzyskane wyniki. 

Praca musi zawierać elementy pracy własnej autora adekwatne do jego wiedzy praktycznej uzyskanej w
okresie studiów. Za pracę własną autora można uznać np.: stworzenie aplikacji informatycznej lub jej
fragmentu, zaproponowanie algorytmu rozwiązania problemu szczegółowego, przedstawienie projektu 
np.~systemu informatycznego lub sieci komputerowej, analizę i ocenę nowych technologii lub rozwiązań
informatycznych wykorzystywanych w przedsiębiorstwach, itp. 

Autor powinien zadbać o właściwą dokumentację pracy własnej obejmującą specyfikację założeń i 
sposób realizacji poszczególnych zadań
wraz z ich oceną i opisem napotkanych problemów. W przypadku prac o charakterze 
projektowo-implementacyjnym, ta część pracy jest zastępowana dokumentacją techniczną i użytkową systemu. 

W pracy \textbf{nie należy zamieszczać całego kodu źródłowego} opracowanych programów. Kod źródłowy napisanych
programów, wszelkie oprogramowanie wytworzone i wykorzystane w pracy, wyniki przeprowadzonych
eksperymentów powinny być umieszczone na płycie CD, stanowiącej dodatek do pracy.

\section*{Styl tekstu}

Należy\footnote{Uwagi o stylu pochodzą częściowo ze stron Macieja Drozdowskiego~\cite{mdro}.} 
stosować formę bezosobową, tj.~\emph{w pracy rozważono ......, 
w ramach pracy zaprojektowano ....}, a nie: \emph{w pracy rozważyłem, w ramach pracy zaprojektowałem}. 
Odwołania do wcześniejszych fragmentów tekstu powinny mieć następującą postać: ,,Jak wspomniano wcześniej, ....'', 
,,Jak wykazano powyżej ....''. Należy unikać długich zdań. 

,,Ilość'' i ,,liczba''. Proszę zauważyć, liczba dotyczy rzeczy policzalnych, np.~liczba osób, liczba zadań, procesorów. 
Ilość dotyczy rzeczy niepoliczalnych, np.~ilość wody, energii. Należy starać się wyrażać precyzyjnie, tj.~zgodnie 
z naturą liczonych obiektów.\footnote{(DW) Według wytycznych Rady Języka Polskiego obie formy są dopuszczalne
zarówno do obiektów policzalnych, jak i niepoliczalnych. W tekstach technicznych warto być jednak precyzyjnym.}

Niedopuszczalne są zwroty używane w języku potocznym. W pracy należy używać terminologii informatycznej, która ma 
sprecyzowaną treść i znaczenie. Nie należy używać ,,gazetowych'' określeń typu: 
silnik bazy danych, silnik programu, maszyna skryptowa, elektroniczny mechanizm, mapowanie, string, gdyż nie wiadomo 
co one właściwie oznaczają. 

Niedopuszczalne jest pisanie pracy metodą \emph{cut\&paste}, bo jest to plagiat i dowód intelektualnej indolencji autora.
Dane zagadnienie należy opisać własnymi słowami. Zawsze trzeba powołać się na zewnętrzne źródła. 



\chapter{Zakończenie}

Zakończenie pracy zwane również Uwagami końcowymi lub Podsumowaniem powinno zawierać ustosunkowanie
się autora do zadań wskazanych we wstępie do pracy, a w szczególności do celu i zakresu pracy oraz
porównanie ich z faktycznymi wynikami pracy. Podejście takie umożliwia jasne określenie stopnia
realizacji założonych celów oraz zwrócenie uwagi na wyniki osiągnięte przez autora w ramach jego
samodzielnej pracy.

Integralną częścią pracy są również dodatki, aneksy i załączniki np.~płyty CDROM
zawierające stworzone w ramach pracy programy, aplikacje i projekty.


% All appendices and extra material, if you have any.
\cleardoublepage\appendix%

\chapter{GitHub}
The thesis and the LeapGesture library are publicly available on the github server: \\
\url{https://github.com/kaathleen/LeapGesture-library/} 

\chapter{Składanie dokumentu w systemie \LaTeX}

Po pierwsze to gratulacje --- dobry wybór. W tym rozdziale znajduje się
garść informacji o tym, jak poprawnie składać tekst pracy w systemie \LaTeX{} wraz z 
przykładami, które mają służyć do przeklejania do własnych dokumentów.

\section{Narzędzia}
Pracując pod systemem Windows, polecam:
\begin{itemize}
    \item MikTeX, \url{http://www.miktex.org/},
    \item JEdit, \url{http://www.jedit.org/},
    \item TeXlipse, \url{http://texlipse.sourceforge.net/},
    \item Kile, \url{http://kile.sourceforge.net/},
    \item Ghostview, Ghostscript (podgląd dokumentów PDF bez blokowania pliku):\\
        \url{http://www.cs.wisc.edu/~ghost/}. 
\end{itemize}

Po zainstalowaniu tych narzędzi wystarczy wykonać polecenie \texttt{compile.bat} (który
jest skryptem wsadowym dla Windows). Dla tych, którzy wolą nieco automatyzacji --- skrypt
\texttt{latexmk}, który jest w MikTeXu (a który potrzebuje zainstalowanego Perla) jest
również bardzo wygodny: \texttt{latexmk -pdf -pvc main.tex}.

\section{Edycja tekstu}
\chaptermark{Tytuł rozdziału, jeśli pełen się nie mieści\ldots{}}{}

\subsection{Struktura dokumentu}

Praca składa się z rozdziałów (\texttt{chapter}) i podrozdziałów (\texttt{section}).
Ewentualnie można również rozdziały zagnieżdzać (\texttt{subsection}, \texttt{subsubsection}),
jednak nie powinno się wykraczać poza drugi poziom hierarchii (czyli \texttt{subsubsection}).

\subsection{Akapity i znaki specjalne}

Każdy akapit to po prostu blok tekstu. Nieważne jak sformatowany --- to zrobi już
system $\LaTeX$.

Akapity rozdziela się od siebie przynajmniej jedną pustą linią. Podstawowe
instrukcje, które się przydają to \emph{wyróżnienie pewnych słów}. Można również
stosować \textbf{styl pogrubiony}, choć nie jest to generalnie zalecane.

Należy pamiętać o zasadach polskiej interpunkcji i ortografii. Po spójnikach 
jednoliterowych warto wstawić znak tyldy ($\sim$), który jest tak zwaną
,,twardą spacją'' i powoduje, że wyrazy nią połączone nie będą rozdzielane
na dwie linie tekstu.

Polskie znaki interpunkcyjne różnią się nieco od angielskich: to jest ,,polski'', a to jest
``angielski''. W kodzie źródłowym tego tekstu będzie widać różnicę.

Proszę również zwrócić uwagę na znak myślnika, który może być pauzą ,,---'' lub
półpauzą: ,,--''. Należy stosować je konsekwentnie. Do łączenia wyrazów używamy
zwykłego ,,-'' (\emph{północno-wschodni}), do myślników --- pauzy lub półpauzy.
Inne zasady interpunkcji i typografii można znaleźć w słownikach.

\subsection{Wypunktowania}

Wypunktowanie z cyframi:
\begin{enumerate}
    \item to jest punkt,
    \item i to jest punkt,
    \item a to jest ostatni punkt.
\end{enumerate}

\noindent
Po wypunktowaniach czasem nie warto wstawiać wcięcia akapitowego. Wtedy przydatne jest
polecenie \texttt{noindent}. Wypunktowanie z kropkami (tzw.~\emph{bullet list}) wygląda tak:
\begin{itemize}
    \item to jest punkt,
    \item i to jest punkt,
    \item a to jest ostatni punkt.
\end{itemize}

\noindent
Wypunktowania opisowe właściwie niewiele się różnią:
\begin{description}
    \item[elementA] to jest opis,
    \item[elementB] i to jest opis,
    \item[elementC] a to jest ostatni opis.
\end{description}


\subsection{Polecenia pakietu \texttt{ppfcmthesis}}

Parę poleceń zostało zdefiniowanych aby uspójnić styl pracy. Są one przedstawione poniżej
(oczywiście nie trzeba się do nich stosować).

\paragraph{Makra zdefiniowane dla języka angielskiego.} Są nimi: \texttt{termdef} oraz \texttt{acronym}.
Przykłady poniżej obrazują ich przewidywane użycie w tekście.
\begin{center}\footnotesize%
\begin{tabular}{l >{\rightskip\fill}p{12cm}}
\toprule
źródło   & \texttt{we call this a $\backslash$termdef\{Database Management System\} ($\backslash$acronym\{DBMS\})} \\ \cmidrule(lr){2-2}
docelowo & we call this a \termdef{Database Management System} (\acronym{DBMS}) \\ 
\bottomrule
\end{tabular}
\end{center}

\paragraph{Makra zdefiniowane dla języka polskiego.} Podobnie jak dla języka angielskiego zdefiniowano
odpowiedniki polskie: \texttt{defini\-cja}, \texttt{akronim} oraz \texttt{english} dla tłumaczeń angielskich
terminów. Przykłady poniżej obrazują ich przewidywane użycie w tekście.
\begin{center}\footnotesize%
\begin{tabular}{l >{\rightskip\fill}p{12cm}}
\toprule
źródło   & \texttt{nazywamy go $\backslash$definicja\{systemem zarządzania bazą danych\} ($\backslash$akronim\{DBMS\}, $\backslash$english\{Database Management System\})} \\ \cmidrule(lr){2-2}
docelowo & nazywamy go \definicja{systemem zarządzania bazą danych} (\akronim{DBMS}, \english{Database Management System}) \\ \bottomrule
\end{tabular}
\end{center}


\subsection{Rysunki}

Format wstawianych rysunków zależy od tego czy używa się do kompilacji polecenia
\texttt{latex}, czy też \texttt{pdflatex}. Oba powinny dać dokładnie ten sam wynik końcowy,
ale praca z nimi jest nieco inna.

\begin{description}
    \item[latex] To polecenie kompiluje źródła \LaTeX{}owe do pliku 
        z rozszerzeniem \texttt{dvi}. Ten plik można przeglądać przy pomocy specjalizowanych programów
        takich jak przykładowo Yap obecny z dystrybucją Mik\TeX{}a. Aby uzyskać docelowy plik \akronim{PDF}
        należy przekonwertować plik \texttt{dvi} przy pomocy programu \texttt{dvipdfm}. 
        
        \textbf{UWAGA:} korzystając z programu \texttt{latex}, wszystkie rysunki muszą być w formacie \akronim{EPS}
        (\english{encapsulated postscript}).

    \item[pdflatex] To polecenie kompiluje źródła \LaTeX{}owe bezpośrednio do pliku \akronim{PDF}.
    
        \textbf{UWAGA:} korzystając z programu \texttt{pdflatex}, wszystkie rysunki muszą być w formacie \akronim{PDF},
        \akronim{JPG} lub \akronim{PNG}.
\end{description}

Można oczywiście używać obu systemów --- wtedy pliki rysunków muszą po prostu być dostępne w obu formatach.

Wszystkie rysunki (w tym również diagramy, szkice i inne) osadzamy w środowisku 
\texttt{figure} i umieszczamy podpis \emph{pod} rysunkiem, w formie elementu \texttt{caption}. Rysunki powinny
zostać umieszczone u góry strony (osadzone bezpośrednio w treści strony zwykle utrudniają czytanie tekstu).
Rysunek~\ref{rys:plama} zawiera przykład pełnego osadzenia rysunku na stronie.

\begin{figure}[t] % możliwe opcje to 't' - top, 'b' - bottom, 'h' - 'here', ale zaleca się 't'
\centering\includegraphics[width=5cm]{figures/template/logo-pp}
\caption{Logo Politechniki Poznańskiej.}\label{rys:plama}
\end{figure}

\begin{figure}[t]
\centering\includegraphics[width=5cm]{figures/template/logo-pp}
\fcmfcaption{Logo Politechniki Poznańskiej. Formatowanie zgodne z wytycznymi FCMu.}\label{rys:plama2}
\end{figure}

Zasady FCMu sugerują nieco inne nagłówki rysunków. Dostepne są one poleceniem \texttt{fcmfcaption} (zob.~rysunek
\ref{rys:plama2}), jeśli ktoś woli mieć podpisy niespójne z rysunkami\ldots

\subsection{Tablice}

Tablice to piękna rzecz, choć akurat ich umiejętne tworzenie w \LaTeX{}u nie jest łatwe. 
Jeśli tablica jest skomplikowana, to pewnie łatwiej będzie ją wykonać w programie
OpenOffice, a następnie wyeksportować jako plik \akronim{PDF}. W każdym przypadku tablice wstawia się podobnie
jak rysunki, tylko że w środowisko \texttt{table}. Tradycja typograficzna sugeruje umieszczenie opisu tablicy, a więc
elementu \texttt{caption} ponad jej treścią (inaczej niż przy rysunkach).  

Tablica~\ref{tab:tabela} pokazuje pełen przykład.

\begin{table}[h]
\caption{Przykładowa tabela. Styl opisu jest zgodny z rysunkami.}\label{tab:tabela}
\centering\footnotesize%
\begin{tabular}{l c}
\toprule
artykuł & cena [zł] \\
\midrule
bułka   & $0,4$ \\
masło   & $2,5$ \\
\bottomrule
\end{tabular}
\end{table}

Zasady FCMu sugerują nieco inne nagłówki tablic. Dostepne są one poleceniem \texttt{fcmtcaption} (zob.~tablicę
\ref{tab:tabela2}), jeśli ktoś woli mieć podpisy niespójne z rysunkami\ldots

\begin{table}[h]
\fcmtcaption{Przykładowa tabela. Styl opisu jest zgodny z wytycznymi FCMu.}\label{tab:tabela2}
\centering\footnotesize%
\begin{tabular}{l c}
\toprule
artykuł & cena [zł] \\
\midrule
bułka   & $0,4$ \\
masło   & $2,5$ \\
\bottomrule
\end{tabular}
\end{table}


\subsection{Checklista}

W katalogu źródeł stylu \texttt{ppfcmthesis} znajduje się plik \texttt{CHECKLIST} --- należy
sprawdzić, czy nie popełniło się któregoś z wymienionych tam błędów.


\section{Literatura i materiały dodatkowe}

Materiałów jest mnóstwo. Oto parę z nich:
\begin{itemize}
    \item \emph{The Not So Short Introduction\ldots}, która posiada również tłumaczenie 
    w języku polskim.\\
    \url{http://www.ctan.org/tex-archive/info/lshort/english/lshort.pdf}

    \item Klasy stylu \texttt{memoir} posiadają bardzo wiele informacji o składzie tekstów
    anglosaskich oraz sposoby dostosowania \LaTeX{}a do własnych potrzeb.\\
    \url{http://www.ctan.org/tex-archive/macros/latex/contrib/memoir/memman.pdf}
    
    \item Nasza grupa dyskusyjna i repozytorium SVN są również dobrym miejscem aby zapytać
    (lub sprawdzić czy pytanie nie zostało już zadane).\\
    \url{https://ophelia.cs.put.poznan.pl/svn/put-latex/trunk}

    \item Dla łaknących więcej wiedzy o systemie \LaTeX{} podstawowym źródłem informacji
    jest książka Lamporta~\cite{Lamport:LDP85}. Prawdziwy \emph{hardcore} to oczywiście
    \emph{The \TeX{}book} profesora Knutha~\cite{Knuth:ct-a}.
\end{itemize}



% Bibliography (books, articles) starts here.
\bibliographystyle{plalpha}{\raggedright\sloppy\small\bibliography{bibliography}}

% Colophon is a place where you should let others know about copyrights etc.
\ppcolophon

\end{document}
