\chapter{Introduction to gesture recognition}\label{introGRChapter}

\section{Taxonomies of gestures proposed in literature}
The vast multiplicity of gestures that human can perform makes the number of classes to which we can divide these gestures substantial. Therefore the classification can be performed in different ways, taking into account different characteristics of gestures. The majority of presented theories include the knowledge that originates from a variety of science such as anthropology, linguistics, cognitive science and other. In this Section, review of the most common gesture classifications in Human Computer Interaction (HCI) context is provided. It is focused mainly on gestures that relate to hand and arm movements.

The basic classification of gestures is the division into static and dynamic gestures. Group of static gesture includes fixed gestures which are not take into account the changes in time. Dynamic gestures is group of time varying gestures.

There is also another general division considered by Kammer et al. \cite{kammer_taxonomy_2010} due to the type of actions activated by the gesture, to online and offline. The first group includes gestures which are processed during performing. It is the group of direct manipulation gestures that provide additional information about dynamics of gesture. They are often used to manipulate objects in space. The second one is the group of action gestures, which are processed at the end of gesture. Most often these are gestures that convey the occurrence of specific meaning.

Karam and schraefel \cite{Karam05ataxonomy} proposed more extensive gesture taxonomy dedicated for Human Computer Interaction. Their classification is based on Quek et al. \cite{Quek:2002:MHD:568513.568514} publication, which provides clarification of gestures taxonomies presented in the past literature. Karam and schraefel defined following gesture classes: deictic, gesticulation, manipulation, semaphores and sign language. In another publication, Aigner et al. \cite{AignerTaxonomy} presented classification, which are more tailored for hand gesture recognition purposes, drawing on concepts from Karam and schraefel work. They distinguished five categories of gestures:

\begin{figure}[htb]
\centering
 \includegraphics[width=0.5\columnwidth]{figures/gestureTypes.png}
 \caption[]{The classification of gestures proposed by Aigner et al. \cite{AignerTaxonomy}}
 \label{gesturetypes}
\end{figure}

\begin{itemize}

\item pointing -- used to point object or indicate direction. Formally, it involve pointing to establish the identity or spatial location of an object within the context of the application domain \cite{Karam05ataxonomy}. This applies not only to indicate by the index finger, but also by any finger or any number of fingers. It is also independent of finger orientation and curvature, while gesture has a indication meaning. Equivalent to deictic gesture in Karam and Schraefel literature.

\item semaphoric -- group which consists of gesture posture and dynamics of gesture, which are used to convey specific meanings. Formally, semaphoric approaches may be referred to as ``communicative'' in that gestures serve as a universe of symbols to be communicated to the machine \cite{Quek:2002:MHD:568513.568514}. Due to the fact that semaphoric are symbolic gestures, their layout can be unrelated to their meaning. Distinguished three semaphoric gesture types of static, dynamic and strokes. The first one concerns specific hand posture, such as thumbs-up meaning approval symbol.
Dynamic semaphorics convey their meaning through movement, for example waving of hand to greet somebody. The last one group are similar to dynamic semaphorics gesture, but this represents fast, stroke-like movements, such as swipe gesture.

\item iconic -- used to demonstrate shape, size, curvature of object or entities. In contrast to the semaphoric gestures, their layout or motion path is strictly related to their meaning. Iconic gestures can be divided into static and dynamic. The former group is performed by hand postures, such as rectangle formed by the thumb and forefingers of both hands. The latter group is often used to map edge line of objects by means of motion paths. For instance showing a simplified sine function characteristics with finger movements.

\item pantomimic -- used to imitate performation of specific task or activity without any tools or objects. Pantomimic gesture are characterized by a high variability of posture and movements. An example of this gesture type can be weapon reload or movement of a knife slicing bread.

\item manipulation -- used to control the position, rotation and scale of the object or entity in space. Manipulation gestures constitute a direct interaction between the manipulated object and a hand or a tool that performs gesture. It follows, that the movement of the manipulated object must be strictly dependent on the motion gesture.
\end{itemize}

\section{State of the art methods}
In this Section, review of the state-of-the-art in human gesture recognition is presented. The problem of gesture recognition can be divided in two main problems: the gesture representation problem and the decision/inference problem. Therefore, this review includes discussion about enabling technology, gesture representations and analysis of recognition methods. Additionally, general problems related to the recognition of gestures and their common solutions are introduced.

\subsection{Sensors}
In this Subsection, the sensors for gesture recognition are overviewed. The main existing gesture recognition approaches related to type of the devices are as follow:
\begin{itemize}
\item non-vision-based devices -- tracking devices, instrumented gloves, armbands and others,
\item vision-based devices -- using one or many cameras.
\end{itemize}

\subsubsection{Non-vision-based Sensors}

This type of devices use various technologies to detect motions, such as accelerometers, multi-touch screens, EMG sensors and other, which include different types of detectors. There are few categories of non-vision-based devices \cite{kaaniche2009human}:

\begin{itemize}

\item wearable -- this kind of device is in the form of garment, which includes sensors needed to recognize arrangement and motions of examined part of the body. It often occur in the form of gloves (CyberGlove®), armband (Myo) or the whole outfit (IGS-190). For instance, CyberGlove® device was used in the system developed by Kevin et al. \cite{KevinCyberGloves}, which recognizes multi-dimensional gestures using condensation-based trajectory matching algorithm. These devices are often related to biomechanical and inertial technologies, 

\item biomechanical -- type of device, which use biomechanical techniques such as electromyography, to measure parameters of gesture. An example of using this type of device is the project developed by Kim et al. \cite{Kim:2008:EHG:1378773.1378778} for Realtime Biosignal Interfacing based on EMG sensors. Example of these devices is Myo armband, which detects gestures and movements using EMG sensors,

\item inertial -- these devices measure the variation of the earth magnetic field in order to detect the motion. This kind of devices use accelerometers \cite{LiuAccelerometer} and gyroscopes \cite{TUD-CS-2009-0292} to measurements,

\item haptics -- various kinds of touch screens. For instance, Webel et al. \cite{conf/vrst/WebelKZ08} developed module for dynamic gestures recognition in multi-touch devices,

\item electromagnetic -- these devices measure the variation of an artificial electromagnetic field generated by wireless networks, electronic devices or produced by themselves. An example of such devices is WiSee, which leverages ongoing wireless transmissions in the environment (e.g., WiFi) to enable whole-home sensing and recognition of human gestures \cite{Pu:2013:WGR:2500423.2500436}.

\end{itemize}

\subsubsection{Vision-based Sensors}

Vision-based sensors include one or several cameras and provide perfomed data from the captured video sequences. Processing of frame is based on filtering, analyzing and data interpreting. The following types of vision-based technology can be distinguished \cite{kaaniche2009human}\cite{Wu:1999:VGR:647591.728702}:

\begin{itemize}

\item typical video cameras -- gesture recognition techniques based on data derived from monocular camera using detection methods such as color or shape based techniques, learning detectors from pixel values or 3D model-based detection,

\item stereocameras -- techniques based on captured images from two cameras, which provide an approximation of the recorded data to a 3D model representation,

\item active techniques -- require the projection of some form of structured light. Examples of this kind devices are Kinect or Leap Motion,

\item invasive techniques -- systems which require using of body markers such as color gloves \cite{Wang:2009:RHC:1531326.1531369}, LED lights (Play Station Move controller).

\end{itemize}

\begin{figure}[htb]
\centering
 \includegraphics[width=1\columnwidth]{figures/gestureRepresentations.png}
 \caption[]{Diagram of gesture representation}
 \label{gesturerepresentations}
\end{figure}

\subsection{Gesture representation}
In this Subsection, overview the spatial modeling of gestures is provided. In particular, greater importance is placed on one type of spatial gesture representations, called the 3D model-based. Depending on the type of the input data, the approach for recognize gesture could be done in different ways.
There are following two main types of gesture representation defined in literature \cite{Huang95handgesture}\cite{Pavlovic97visualinterpretation}:
\begin{itemize}
\item 3d model-based,
\item appearance-based.
\end{itemize}

\subsubsection{3D model-based}

Defines the 3D spatial decription of the human body parts. They can be classified in two large groups:
\begin{itemize}
\item volumetric models,
\item skeletal models.
\end{itemize}
Volumetric models reproduce with high accuracy the shape of the hand or arm. Real object is often interpreted as mesh of vertices or non-uniform rational B-spline (NURBS) ordered in space. This model is commonly used for computer vision purposes or for computer animation. The drawback of this approach is that, it is computationally demanding and difficult to analyze in real time. Simplified models of the hand or arm with the reduced set of parameters (such as skeletal models) are often used \cite{Pavlovic97visualinterpretation}.

As indicated earlier, instead of dealing with all the parameters of volumetric type, model can be reduced to set of equivalent joint angle parameters together with segment lengths. These are known as skeletal models. There are several advantages of using this representation type:
\begin{itemize}
\item Simplified model with the most important parameters allows the detection program to focus on the significant parts of the body.
\item Due to the smaller amount of data, the processing algorithms are faster.
\end{itemize}

\subsubsection{Appearance-based}

The second group of models do not use direct description of the spatial object points, because this model is based on shape of hands or arms in the visual images. The gestures are modeled by relating the appearance of any gesture to the appearance of the set of predefined, template gestures \cite{Pavlovic97visualinterpretation}. In this group distinguished a large variety of models. The most used 2D models are:
\begin{itemize}
\item color based model -- in general, using body markers to track the motion of the body part,
\item binary silhouette based model -- models based on the geometric properties of the object silhouette,
\item deformable gabarit based model -- they are generally based on deformable active contours,
\item motion-based model -- based on the motion of individual pixels or image part description.
\end{itemize}

\subsection{Gesture recognition methods}\label{DynamicStaticText}
As it was written earlier, one of the main issues of gesture recognition is the decision problem. Currently several solutions have been proposed, which can be used regardless of device type or data representation for different classes of gestures. Classification of gestures, which should be taken into account while choosing gesture recognition method, are static and dynamic division. For each of them may be used different tools due to the different properties of these gestures. In the case of static gesture recognition, an important feature is arrangement of object, which performs a gesture, in other words, how the individual parts of the object are arranged in relation to each other. For dynamic gestures --- as described in the previous Subsection --- a very important feature is the variation in time (dynamics of the gesture or dynamics of individual parts of the object performing the gesture).

To recognize static gestures, general classifier, neural network or template-matcher can be used. Methods which are capable to recognize dynamic gestures have to take into account the aspect of time. An example of this kind of method is Hidden Markov Model.

\subsubsection{Static gesture recognition}

Different techniques to perform accurate static gesture recognition have been decribed in the literature. The most common methods are neural networks, support vector machines and simple pattern techniques \cite{journals/jbcs/SavarisW10}.

\emph{A neural network (NN)} is an information-processing system that has been developed as generalizations of mathematical models of human cognition or neural biology. A neural network is characterized by its pattern of connections between the neurons, method of determining the weights on the connections, and its activation function \cite{Fausett:1994:FNN:197023}. They can be used both for static and dynamic gestures.

Hasan et al. \cite{HasanStaticHand} presented hand gesture recognition based on shape analysis. Tests were conducted for six static gestures using multi-layer perception of neural network and back-propagation learning algorithm. NN architecture consisted of one hidden layer ($100$ nodes), $1060$ inputs and $6$ output for each gesture. They achieved a recognition rate of $86.38\%$ for a training set of $30$ images and a testing set of $84$ images.

Xu et al. \cite{conf/icat/XuYZ06} developed virtual training system of Self-Propelled Gun based on static gesture recognition and hand translations and rotations. The input data for algorithms was captured using a $18$-sensor DataGlove. To recognize gestures was used feed-forward neural network with $40$ nodes in single hidden layer, $18$ input and $15$ output nodes. The back-propagation using a variable learning rate is selected as training method. The tests were conducted on a set of $300$ hand gestures from five different people --- $200$ gestures for training set and $100$ for testing set. With the use of these methods the authors reached gesture recognition performance of $98\%$.

In publication of Stergiopoulou and Papamarkos publication \cite {Stergiopoulou:2009:HGR:1651923.1651954} can be found static gesture recognition through other type of neural network --- Self-Growing and Self-Organized Neural Gas (SGONG). To quote the authors, SGONG is innovative neural network that grows according to the morphology of hands in a very robust way. The algorithms were tested for $31$ hand gestures that derive from the combination of shown and hidden fingers. Data were collected from a camera, and the recordings of hand were created in a vertical position on uniform background. With these assumptions, gesture recognition rate of $90.45\%$ have been reached but required average computation time was about $1.5$ s, using a $3$ GHz CPU.

\emph{Support-Vector Machine (SVM)} is a classification method invented by Vapnik \cite{Cortes:SVM}. SVM is a supervised learning algorithm used for classification and regression analysis, based on the mapping of characteristics extracted from instances namely the feature vectors to points in space. SVM constructs in multi-dimensional space, set of hyperplanes, which non-linearly divide points in this space (input vectors) to different classes. Support Vector Machines can be called a maximum margin classifier, because the resulting hyperplanes maximize the distance between the ’nearest’ vectors of different classes. These ``nearest'' vectors are called support vectors.

Chen and Tseng \cite{ChenDeveloping} presented system based on training SVM which allows effective recognition gesture in popular game, rock-paper-scissors. One of the challenges of their work was to teach the classifier to recognize multiple-angle hand gesture. The collection of training and testing data were images from video camera, which are preprocessed using convertion to grayscale and histogram equalization. Data were collected from $5$ different people for the right hand only. For the learning set consisted of $420$ images and testing set of $120$ images, the recognition rate of $95\%$ was achieved.

Rahman and Afrin \cite{RahmanHand} presented hand gesture recognition system which recognizes static hand gesture for alphabet of $10$ letters using Biorthogonal Wavelet Transform and SVM. Input data in the form of images --- in addition to filtering --- are transformed by the Canny edge detection method and then processed sequentially through Radon and Biorthogonal Wavelet Transformations. Finally, the data in this form are transmitted to the SVM classifier. To achieve robustness of the method to varying conditions, authors used a large dataset --- $800$ positive samples and $1500$ negative image samples. Average recognition rate was $87.4\%$.

Liu et al. \cite{LiuStatic} proposed recognition method based on SVM and Hu moments which applied to Chinese Driver Physical Examination System. For collection of $2416$ positive samples and $3050$ negative samples from $20$ people recognition rate of $96.5\%$ have been reached.

Ren and Zhang \cite{RenMEBSVM} proposed other recognition method named by them as MEB-SVM. This method combines the SVM with minimum enclosing ball (MEB) and --- according to the authors --- allows to reduce computation with effective separation of all kinds of vectors in hyperspace. The input data used to test this method are images which are initially binarized and then countour line is retrieved. Finally, contour line is converted by means of Fourier transform, so that data are independent of translation, rotating and zooming. Their method achieved a recognition rate of $92.9\%$.


Dominio et al. \cite{Dominio:2013:HGR:2510650.2510651} presented novel hand gesture recognition based on depth data using Kinect device. The proposed processing consists following, main steps: extraction hand region from the depth map and subdivided it into palm and finger samples, extraction set of features based on finger tips and center of the hand, classification by SVM. Based on $1,000$ different depth maps with $10$ gestures performed by $10$ different people, they achieved mean recognition rate of $99.5\%$.

Other popular methods are \emph{simple pattern recognition techniques}. This group includes methods based on a simple comparison the characteristics of new problem instance with instances seen in training, instead of performing explicit generalization. In the case of gesture recognition, output information of algorithm is evaluated on the basis of similarity of the gesture to other pre-defined or learned gestures, for which belonging to groups is known. Basis on this, it is concluded that the newly read gesture belongs to the group. These techniques are generally based on a efficient lazy learning methods such as instance-based learning methods.  In the context of gesture recognition, the most widely used algorithm is the k-nearest neighbour.

Ziaie et al. \cite{ziaie-etal-csicc2008-revised} combine naïve Bayes classifier with k-nearest neighbors algorithm for static gesture recognition purposes. Based on the dataset from the camera consisting of $580$ samples for the $3$ gestures, authors have achieved $93\%$ of recognition rate.

Chang et al. \cite{journals/jise/ChangCTH06} proposed new approach for static gesture recognition based on Zernike moments (ZMS) and pseudo-Zernike moments (PZMs), which provide greater independence from the translation of gestures. For gesture classification, k-nearest neighbor has been used, which directly processes the data from the ZMS and PZMs. In addition, authors used a minimum bounding circle, which supports the decomposition of hand silhouette into the finger parts and palm part. For dataset of $600$ samples with $6$ gestures performed by $10$ people, gesture recognition rate of $97.3\%$ has been reached.

\subsubsection{Dynamic gesture recognition}

Referring to the previously proposed classification of gestures may be considered that dynamic gestures include a wide range of existing types of gestures and have an important role in interpersonal communication as well as in HCl. Therefore, many approaches were proposed to gesture recognition taking into account the temporal aspect of gestures. Shen et al. \cite{Shen:2012:DHG:2206425.2206457} propose following division of these approaches:

\begin{itemize}
\item model-based methods,
\item exemplar-based methods.
\end{itemize}

The first one includes methods that are based on the 3D model-based gesture representation, and which assume that the hand was detected and hand movements are being tracked. Model-based approaches include Hidden Markov Models (HMM), Finite State Machines (FSM), Time Delay Neural Networks (TDNN) and self-organizing networks, which preserve topology.

Exemplar-based is group of methods, which performs recognition by exemplar or template matching. These methods use a visual representation of data such as spatio-temporal local features, motion trajectiories, bag-of-features representation and other. However, this approach of dynamic gesture recognition will not be considered in thesis.

First model-based method is \emph{Hidden Markov Models}. HMM is considered as specific form of dynamic Bayesian network and is doubly stochastic process with an underlying stochastic process that is not observable (it is hidden), but can only be observed through another set of stochastic processes that produce the sequence of observed symbols \cite{RabinerHmmIntro}. This model is represented by set of finite states connected by transitions. Each state is characterized by the state transition probabilities and probability distribution of emitting output tokens. HMM is one of the most commonly used methods for dynamic gesture recognition.

The use of HMM in the context of gesture recognition has been proposed by Yamato el al. \cite{YamatoComputerVision}. In this approach, a discrete HMM and image feature vector sequence converted to symbolic sequences by vector quantization have been used to recognize six classes of tennis strokes.

Yang and Xu \cite{Yang_1994_329} proposed use of multi-dimensional Hidden Markov Model for handdrawn gestures recognition purposes. For tests, $9$ gestures representing numbers from $1$ to $9$ was defined. Based on the $6$-states Bakis model, the training set consisting of $100$ samples and a test set of $450$ samples, achieved recognition rate $99.78\%$.

In publication of Starner and Pentland \cite{StarnerComputerVision} HMM was used for real-time recognizing sentence-level American Sign Language. The authors achieved a processing speed of 5 frames per second with recognition rate of $99.2\%$ for $494$ sentences.

In recent work also proposed new approaches and improvements to HMM method, such as using semantic network model (SNM) which introduces semantic states \cite{RajkoComputerVision}, hidden state conditional random field model \cite{WangComputerVision} and non-parametric HMM for exemplar-based gesture recognition \cite{Elgammal:2003:LDE:1965841.1965916}.

\emph{Finite State Machines} or Finite State Automation are computational models that allow to design sequential logic circuits. FSM consist a finite number of states, which one of them is marked as the current state. The transition between the states is performed in the case of occurrence of the corresponding event. FSM is also defined by a states set and collection of transition conditions.

Davis and Shah \cite{Davis94visualgesture} used FSM for hand gesture recognition using model based approach. In their approach, fingers tracking has been applied to determine the motion paths and to find the start and stop positions of gestures. Gesture was modelled as set of vectors with motion key, using motion correspondence algorithm.

Hong et al. \cite{Hong00constructingfinite} also applied FSM for gesture recognition using sequences of states in spatial-temporal space as gestures model. Each state is represented by multidimensional Gaussian.

Another method is \emph{Time Delay Neural Networks} (TDNN), which is a group of neural networks with a special architecture. The main feature of this approach is that recognition is independent of the time offset of the sample. TDNN operates on sequential data.

Yang and Ahuja \cite{YangAhujaComputerVision} applied TDNN to recognize American Sign Language (ASL), just like Starner and Pentland in the aforementioned publication. They provided extracting and classifying of two-dimensional dynamic gestures. Using motion trajectories, multiscale segmentation and affine transformations authors achieved $96.21\%$ for testing set of ASL gestures.

The last approach is \emph{self-organizing networks}, which are characterized by automatic adapting to the input data. This allows to dynamically adapt the network to new data.

Florez et al. \cite{Florez:2002:HGR:874061.875461} presented approach based on self-organizing neural network, which are capable of characterizing hand posture as well as movements. Their method reached recognition rate of $97.08\%$ for $12$ gestures.
