
\chapter{Abstract}
Nowadays, when computers are ubiquitous, people need to develop more natural and intuitive interface to use computer. People would like to use gestures of everyday life and translate them into a virtual world. This paper presents library for gesture recognition dedicated to Leap Motion Controller called LeapGesture. Leap Motion is a new device, which tracks fingers and other objects up to 1/100th of a millimeter. The LeapGesture library can be used for various kinds of gestures. This can be done, because of the use of different detection methods for different types of gestures. For static gestures recognition has been used support vector machine (SVM) and for dynamic gestures -- hidden Markov model (HMM). For static gestures was achieved recognition accuracy of 93\% using SVM for 15 gestures. The recognition rate for dynamic gestures reached 80\% using HMM. In library can be found additional modules: recorder -- for recording gestures in a format supported by the library, visualizer -- for reviewing recorded gestures and finger differentiation module, which recognizes the straightened fingers. To differentiate fingers has been used SVM. In this case the recognition rate reached 93\%. In this paper has been also presented classification of gestures and its modification in the context of Leap Motion Controller. Additionally this work includes descriptions of helper methods used for the gesture recognition, such as pre-processing of data obtained from the device, which gets rid of existing noise or method for fingers differentiating, whereby the obtained results are more accurate. 


% Pojechalem troche po abstrakcie, bo wedlug mnie jest za duzo lania wody, a za malo informacji, co tam naprawde zostalo zrobione ;) oraz troche za duzo ---

% Tu macie moj abstract z poprzedniej inz - jest pare jezykowych wpadek, ale 
% chodzi mi raczej o strukture

%The aim of this BSc thesis is to investigate the state-of-the-art methods
%of mobile robot self-localization using RGB-D data from the Kinect sensor,
%and to propose some improvements to these methods. The author of this thesis
%implemented five methods using various feature detectors, descriptors and
%various algorithms for estimation of the sensor motion. Three new methods
%are proposed in this thesis to address known drawbacks of the state-of-the-art
%algorithms. Experiments were conducted in order to measure the localization
%precision and time of processing of the methods being investigated. The methods
%were compared on datasets from the Poznan University of Technology
%and on datasets from the Freiburg University. The experiments allowed to
%identify the advantages and drawbacks of the methods. As a result, one of the
%proposed method was chosen as the fastest and most realiable. In the future
%this method can be used as a part of a full SLAM system for a mobile robot.

Since the invention of computers there exists a need to develop more intuitive human-computer interfaces.
There has been keyboards, mouses, but those solutions are not as natural as gestures, which are essential part of a human expression.
This thesis studies the new possibilities to gesture interfaces that emerged with a Leap Motion sensor.
The Leap Motion is an innovative, 3D motion capturing device designed especially for hands and fingers tracking with precision up to 0.01mm.
The authors examined the sensor's data and possible usage as the gesture interface utilizing two types of gestures: static and dynamic.
The static gestures understood as poses of a hand and fingers are recognized using the Support Vector Machine (SVM) with intelligent pre- and postprocessing.
The proposed approach allowed to recognize five gestures with 99\% accuracy and ten gestures with 85\%.
The dynamic gestures that are movements of a hand and fingers in time are recognized with the Hidden Markov Models. 
The adopted approach allowed to achieve accuracy up to 80\% for five dynamic gestures.
The thesis contains also experiments of the proposed finger differentiation module working with x\% accuracy.
The main outcome of the thesis is the open-source library dedicated to the developers, that contains presented approaches using a high-level C++ interface making gesture recognition easy to utilize in any application.