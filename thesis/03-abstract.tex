
\chapter{Abstract}
Nowadays, when computers are ubiquitous, people need to develop more natural and intuitive interface to use computer. People would like to use gestures of everyday life and translate them into a virtual world. This paper presents library for gesture recognition dedicated to Leap Motion Controller called LMGesture.  Leap Motion is a new device, which tracks fingers and other objects up to 1/100th of a millimeter. The LMGesture library can be used for various kinds of gestures. This can be done, because of the use of different detection methods for different types of gestures. For static gestures recognition has been used support vector machine (SVM) and for dynamic gestures -- hidden Markov model (HMM). In library can be found additional modules: recorder -- for recording gestures in a format supported by the library, visualizer -- for reviewing recorded gestures, finger differentiation module -- for differentiating fingers in a performed gesture.  In this paper has been also presented classification of gestures and its modification in the context of Leap Motion Controller. Additionally this work includes descriptions of helper methods used for the gesture recognition, such as pre-processing of data obtained from the device, which gets rid of existing noise or method for fingers differentiating, whereby the obtained results are more accurate.

{\color{red} [dodac opis testow]}